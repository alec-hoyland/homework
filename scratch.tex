\section{The Lost Laboratory of Viveca}

The artificer Viveca was born in what would become Hallowdale in the year
-300 yv. She became apprenticed to an illusionist, Ostrik, upon reaching her majority
and excelled in the craft. However, while her creativity knew little bounds, the
works of illusion did not penetrate into the material. She took up the practice
of artifice in order to supplement her magic, to the consternation of her wizardly
peers.

Around her second century, she found herself in the Plane of Shadow, where she
worked on resurrecting a tradition of shadow conjuration and evocation, using
``shadowstuff'' to enhance illusions into the realm of reality.

She found herself in the elven ruins of Daoine Gloine, where she set about her work.

\paragraph{Viveca Naguma the Shaper.}
Chaotic Neutral female rock gnome illusionist wizard 20.
Standing about three and a quarter feet tall, Viveca is a stout middle-aged rock
gnome with austere features, a patrician nose, creases near her eyes and brows from
squinting, and gray overtaking the auburn from the temples. She is interested only
in her research, and does not care for the plight of the Winterfell or the threat
of the arch-wraith. She is happy to sell her equipment to the adventurers if they
promise to provide technical feedback and not use them for expressly evil purposes.

Viveca has developed several spells reconstructed from the spellbooks of wizards
long since dead.

\spacedlowsmallcaps{Shadow Conjuration} \\
\textit{4th level illusion} \\
\textbf{Casting Time:} 1 action \\
\textbf{Range:} Special \\
\textbf{Components:} V, S \\
\textbf{Duration:} Special \\

You use material from the Plane of Shadow to shape quasi-real illusions mimicking
a conjuration spell on the sorcerer or wizard spell list of 3rd level or lower with
a casting time of 1 action or less.

The mimicked spell takes effect as if you had just cast it at the level of the spell
at which it is being mimicked.

Shadow conjurations are half as strong as they ought to be. A creature that interacts
with the conjured object, force, or creature must make a Wisdom saving throw
to recognize its true nature. On successful save, the subject gains resistance to
all damage dealt by the shadow conjuration and advantage on all saving throws against
its effects. On a failed save, the shadow conjuration appears real to the subject
and can affect it normally.

A creature that succeeds on its save sees the shadow conjurations as transparent
images superimposed on vague, shadowy forms.

Objects automatically succeed on their Wisdom saving throws against this spell.

\textbf{\textit{At Higher Levels:}} When you cast this spell using a spell slot of
5th level or higher, you can mimick a conjuration spell of one level lower with this
spell.

\textbf{\textit{Classes:}} Bard, Sorcerer, Wizard

\spacedlowsmallcaps{Shadow Evocation} \\
\textit{5th level illusion} \\
\textbf{Casting Time:} 1 action \\
\textbf{Range:} Special \\
\textbf{Components:} V, S \\
\textbf{Duration:} Special \\

You use material from the Plane of Shadow to shape quasi-real illusions mimicking
an evocation spell on the sorcerer or wizard spell list of 4th level or lower with
a casting time of 1 action or less.

The mimicked spell takes effect as if you had just cast it at the level of the spell
at which it is being mimicked.

Shadow evocations are half as strong as they ought to be. A creature that interacts
with the conjured object, force, or creature must make a Wisdom saving throw
to recognize its true nature. On successful save, the subject gains resistance to
all damage dealt by the shadow evocation and advantage on all saving throws against
its effects. On a failed save, the shadow evocation appears real to the subject
and can affect it normally.

A creature that succeeds on its save sees the shadow evocations as transparent
images superimposed on vague, shadowy forms.

Objects automatically succeed on their Wisdom saving throws against this spell.

\textbf{\textit{At Higher Levels:}} When you cast this spell using a spell slot of
6th level or higher, you can mimick an evocation spell of one level lower with this
spell.

\textbf{\textit{Classes:}} Bard, Sorcerer, Wizard

\paragraph{Adapting the Adventure.}
This adventure is adapted from \texttt{The Lost Laboratory of Kwalish}. The main
difference is that this adventure is intended for characters of 15th level. This
means that certain creatures will be enhanced and the power of the monks increases
by commensurate amounts.

\paragraph{The Monastery of the Distressed Body}
There are fifty monks instead of twenty-five and some are more powerful. All are
evil humanoids of various races and ethical alignments.

\begin{itemize}
  \item 20 cultists
  \item 5 cult fanatics
  \item 10 galvanic aspirants (treat as \textbf{galvanic blastseekers})
  \item 10 flux aspirants (treat as \textbf{flux blastseekers})
  \item 5 devilbound ascendants (treat as \textbf{devilbound gnomish princes})
  \item The Grand Master, an amnizu devil
\end{itemize}

\section{Cragmaw Chasm}

Where in the material plane there crumbles an ancient fortress, in the Plane
of Shadow, we find a massive temple, raised to some forgotten god or terror.

\subsection{Entrance}
The temple can be entered through the glass gazebo on the lefthand side. The rug on the floor depicts a dark moon blocking out the stars in a perfect disk.

\paragraph{Trap (5 xp).} If any light, magical or nonmagical is brought into this chamber, the creature bearing the light must make a \textbf{Dexterity saving throw DC 18} or take 10d10 radiant damage, as the light refracts off the windows and returns amplified with deadly force.

\subsection{Candelabras}
Inside the temple antechamber stands massive candelabras. The ceiling reaches thirty feet
vaults up with ornate struts carved to resemble stern angels and devils supporting the ceiling. The candelabras are twenty feet high. The righthand side one is unlit but the lefthand side is lit by will-o'-wisps, forced into eternal servitude.

\subsection{Crypt of the Elders}
Two tapestries flank this chamber, its ceiling sloping from 30 feet high to 20 feet, ending in stained glass windows depicting a purple-black dragon with silver ridges, and a woman in armor wielding a shield and longsword. The shield is in the forefront, depicting a sprig of wolfsbane.

\paragraph{Puzzle.}
The woman is Saint Aurica Markovia of the Church of Ezra. The dragon is Arkilorn, a shadow dragon who dwells in the mountains. This battle, historically, never took place, and seems to be the provenance of overzealous artists taking liberty with the legends.

The tapestries to the left and right depict a man and a woman, both elves with dusky features and long dark hair. The man wears robes, as if a wizard, and carries a staff of straight wood embedded with jewels. The woman has her hands above her head clasping a long black-bladed sword that emananates shadowy force.

\paragraph{Puzzle (5 xp).}
These two elves are buried here in their respective sarcophagi. The lids are heavy and sealed with \textit{sovereign glue}, making them exceptionally difficult to open. They can be smashed to bits by dealing 40 or more damage to the lids.

\paragraph{Treasure (5 xp).}
The male elf is buried with a \textit{staff of power}. The female elf is buried with \textit{blackrazor}, a terrible sword of dread portent.

\paragraph{Trap (10 xp).}
Each sarcophagus is trapped with \textit{symbols of fear} and \textit{death} respectively DC 17.

\subsection{Gorgon Lair}
The skull lord has bound a gorgon here. It is afflicted with a 7th level \textit{geas} spell to guard this chamber. It is free to move about the temple complex, but if it strays or shirks its duty, it suffers terrible pain, and so returns to its lair.

\paragraph{Encounter (1 xp).}
The gorgon.

\paragraph{Treasure.}
Some time ago, an adventuring party sought to steal from the temple. They ultimately were unsuccessful, having been killed and eaten by the gorgon. They had attempted to carry their ill-gotten gains in a treasure chest hoisted by magic. Since their untimely demise, the chest has sat in the room where they died.

Their treasure includes a \textit{potion of necrotic resistance}, a \textit{potion of thunder resistance}, a \textit{potion of clairvoyance}, a \textit{scroll of protection from fiends}, a \textit{scroll of maze}, a \textit{scroll of knock} and of \textit{pyrotechnics}, \textit{oil of slipperiness}, \textit{dust of sneezing and choking}, and 4,297 gold pieces.
