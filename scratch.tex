\section{The Lost Laboratory of Viveca}

The artificer Viveca was born in what would become Hallowdale in the year
-300 yv. She became apprenticed to an illusionist, Ostrik, upon reaching her majority
and excelled in the craft. However, while her creativity knew little bounds, the
works of illusion did not penetrate into the material. She took up the practice
of artifice in order to supplement her magic, to the consternation of her wizardly
peers.

Around her second century, she found herself in the Plane of Shadow, where she
worked on resurrecting a tradition of shadow conjuration and evocation, using
``shadowstuff'' to enhance illusions into the realm of reality.

She found herself in the elven ruins of Daoine Gloine, where she set about her work.

\paragraph{Viveca Naguma the Shaper.}
Chaotic Neutral female rock gnome illusionist wizard 20.
Standing about three and a quarter feet tall, Viveca is a stout middle-aged rock
gnome with austere features, a patrician nose, creases near her eyes and brows from
squinting, and gray overtaking the auburn from the temples. She is interested only
in her research, and does not care for the plight of the Winterfell or the threat
of the arch-wraith. She is happy to sell her equipment to the adventurers if they
promise to provide technical feedback and not use them for expressly evil purposes.

Viveca has developed several spells reconstructed from the spellbooks of wizards
long since dead.

\spacedlowsmallcaps{Shadow Conjuration} \\
\textit{4th level illusion} \\
\textbf{Casting Time:} 1 action \\
\textbf{Range:} Special \\
\textbf{Components:} V, S \\
\textbf{Duration:} Special \\

You use material from the Plane of Shadow to shape quasi-real illusions mimicking
a conjuration spell on the sorcerer or wizard spell list of 3rd level or lower with
a casting time of 1 action or less.

The mimicked spell takes effect as if you had just cast it at the level of the spell
at which it is being mimicked.

Shadow conjurations are half as strong as they ought to be. A creature that interacts
with the conjured object, force, or creature must make a Wisdom saving throw
to recognize its true nature. On successful save, the subject gains resistance to
all damage dealt by the shadow conjuration and advantage on all saving throws against
its effects. On a failed save, the shadow conjuration appears real to the subject
and can affect it normally.

A creature that succeeds on its save sees the shadow conjurations as transparent
images superimposed on vague, shadowy forms.

Objects automatically succeed on their Wisdom saving throws against this spell.

\textbf{\textit{At Higher Levels:}} When you cast this spell using a spell slot of
5th level or higher, you can mimick a conjuration spell of one level lower with this
spell.

\textbf{\textit{Classes:}} Bard, Sorcerer, Wizard

\spacedlowsmallcaps{Shadow Evocation} \\
\textit{5th level illusion} \\
\textbf{Casting Time:} 1 action \\
\textbf{Range:} Special \\
\textbf{Components:} V, S \\
\textbf{Duration:} Special \\

You use material from the Plane of Shadow to shape quasi-real illusions mimicking
an evocation spell on the sorcerer or wizard spell list of 4th level or lower with
a casting time of 1 action or less.

The mimicked spell takes effect as if you had just cast it at the level of the spell
at which it is being mimicked.

Shadow evocations are half as strong as they ought to be. A creature that interacts
with the conjured object, force, or creature must make a Wisdom saving throw
to recognize its true nature. On successful save, the subject gains resistance to
all damage dealt by the shadow evocation and advantage on all saving throws against
its effects. On a failed save, the shadow evocation appears real to the subject
and can affect it normally.

A creature that succeeds on its save sees the shadow evocations as transparent
images superimposed on vague, shadowy forms.

Objects automatically succeed on their Wisdom saving throws against this spell.

\textbf{\textit{At Higher Levels:}} When you cast this spell using a spell slot of
6th level or higher, you can mimick an evocation spell of one level lower with this
spell.

\textbf{\textit{Classes:}} Bard, Sorcerer, Wizard

\paragraph{Adapting the Adventure.}
This adventure is adapted from \texttt{The Lost Laboratory of Kwalish}. The main
difference is that this adventure is intended for characters of 15th level. This
means that certain creatures will be enhanced and the power of the monks increases
by commensurate amounts.

\paragraph{The Monastery of the Distressed Body}
There are fifty monks instead of twenty-five and some are more powerful. All are
evil humanoids of various races and ethical alignments.

\begin{itemize}
  \item 20 cultists
  \item 5 cult fanatics
  \item 10 galvanic aspirants (treat as \textbf{galvanic blastseekers})
  \item 10 flux aspirants (treat as \textbf{flux blastseekers})
  \item 5 devilbound ascendants (treat as \textbf{devilbound gnomish princes})
  \item The Grand Master, an amnizu devil
\end{itemize}
