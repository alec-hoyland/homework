\documentclass{article}

\usepackage{fancyhdr}
\usepackage{extramarks}
\usepackage{amsmath}
\usepackage{amsthm}
\usepackage{amsfonts}
\usepackage{tikz}
\usepackage[plain]{algorithm}
\usepackage{algpseudocode}
\usepackage[title={Colophon},titlestyle=itshape,titlesize=24pt, titlealign=r,parstyle=upshape, parsize=10pt,parlead=11pt, noclrdblpg,nofullpage, topspace=2in]{colophon}

\usetikzlibrary{automata,positioning}

%
% Basic Document Settings
%

\topmargin=-0.45in
\evensidemargin=0in
\oddsidemargin=0in
\textwidth=6.5in
\textheight=9.0in
\headsep=0.25in

\linespread{1.1}

\pagestyle{fancy}
\lhead{\hmwkAuthorName}
\chead{\hmwkClass\: \hmwkTitle}
\rhead{\firstxmark}
\lfoot{\lastxmark}
\cfoot{\thepage}

\renewcommand\headrulewidth{0.4pt}
\renewcommand\footrulewidth{0.4pt}

\setlength\parindent{0pt}

%
% Create Problem Sections
%

\newcommand{\enterProblemHeader}[1]{
    \nobreak\extramarks{}{Problem \arabic{#1} continued on next page\ldots}\nobreak{}
    \nobreak\extramarks{Problem \arabic{#1} (continued)}{Problem \arabic{#1} continued on next page\ldots}\nobreak{}
}

\newcommand{\exitProblemHeader}[1]{
    \nobreak\extramarks{Problem \arabic{#1} (continued)}{Problem \arabic{#1} continued on next page\ldots}\nobreak{}
    \stepcounter{#1}
    \nobreak\extramarks{Problem \arabic{#1}}{}\nobreak{}
}

\setcounter{secnumdepth}{0}
\newcounter{partCounter}
\newcounter{homeworkProblemCounter}
\setcounter{homeworkProblemCounter}{1}
\nobreak\extramarks{Problem \arabic{homeworkProblemCounter}}{}\nobreak{}

%
% Homework Problem Environment
%
% This environment takes an optional argument. When given, it will adjust the
% problem counter. This is useful for when the problems given for your
% assignment aren't sequential. See the last 3 problems of this template for an
% example.
%
\newenvironment{homeworkProblem}[1][-1]{
    \ifnum#1>0
        \setcounter{homeworkProblemCounter}{#1}
    \fi
    \section{Problem \arabic{homeworkProblemCounter}}
    \setcounter{partCounter}{1}
    \enterProblemHeader{homeworkProblemCounter}
}{
    \exitProblemHeader{homeworkProblemCounter}
}

\newcommand{\hmwkTitle}{Homework\ \#1}
\newcommand{\hmwkDueDate}{2019-9-19}
\newcommand{\hmwkClass}{CAS CS 640}
\newcommand{\hmwkClassTime}{Section A1/A5}
\newcommand{\hmwkClassInstructor}{Dr. Margrit Betke}
\newcommand{\hmwkAuthorName}{\textbf{Alec Hoyland (U83403624)}}

%
% Title Page
%

\title{
    \vspace{2in}
    \textmd{\textbf{\hmwkClass:\ \hmwkTitle}}\\
    \normalsize\vspace{0.1in}\small{Due\ on\ \hmwkDueDate\ at 3:30pm}\\
    \vspace{0.1in}\large{\textit{\hmwkClassInstructor\ \hmwkClassTime}}
    \vspace{3in}
}

\author{\hmwkAuthorName}
\date{}

\renewcommand{\part}[1]{\textbf{\large Part \Alph{partCounter}}\stepcounter{partCounter}\\}

%
% Various Helper Commands
%

% Useful for algorithms
\newcommand{\alg}[1]{\textsc{\bfseries \footnotesize #1}}

% For derivatives
\newcommand{\deriv}[1]{\frac{\mathrm{d}}{\mathrm{d}x} (#1)}

% For partial derivatives
\newcommand{\pderiv}[2]{\frac{\partial}{\partial #1} (#2)}

% Integral dx
\newcommand{\dx}{\mathrm{d}x}

% Alias for the Solution section header
\newcommand{\solution}{\textbf{\large Solution}}

% Probability commands: Expectation, Variance, Covariance, Bias
\newcommand{\E}{\mathrm{E}}
\newcommand{\Var}{\mathrm{Var}}
\newcommand{\Cov}{\mathrm{Cov}}
\newcommand{\Bias}{\mathrm{Bias}}

\begin{document}

\maketitle

\pagebreak

% \begin{homeworkProblem}
%     Give an appropriate positive constant \(c\) such that \(f(n) \leq c \cdot
%     g(n)\) for all \(n > 1\).
%
%     \begin{enumerate}
%         \item \(f(n) = n^2 + n + 1\), \(g(n) = 2n^3\)
%         \item \(f(n) = n\sqrt{n} + n^2\), \(g(n) = n^2\)
%         \item \(f(n) = n^2 - n + 1\), \(g(n) = n^2 / 2\)
%     \end{enumerate}
%
%     \textbf{Solution}
%
%     We solve each solution algebraically to determine a possible constant
%     \(c\).
%     \\
%
%     \textbf{Part One}
%
%     \[
%         \begin{split}
%             n^2 + n + 1 &=
%             \\
%             &\leq n^2 + n^2 + n^2
%             \\
%             &= 3n^2
%             \\
%             &\leq c \cdot 2n^3
%         \end{split}
%     \]
%
%     Thus a valid \(c\) could be when \(c = 2\).
%     \\
%
%     \textbf{Part Two}
%
%     \[
%         \begin{split}
%             n^2 + n\sqrt{n} &=
%             \\
%             &= n^2 + n^{3/2}
%             \\
%             &\leq n^2 + n^{4/2}
%             \\
%             &= n^2 + n^2
%             \\
%             &= 2n^2
%             \\
%             &\leq c \cdot n^2
%         \end{split}
%     \]
%
%     Thus a valid \(c\) is \(c = 2\).
%     \\
%
%     \textbf{Part Three}
%
%     \[
%         \begin{split}
%             n^2 - n + 1 &=
%             \\
%             &\leq n^2
%             \\
%             &\leq c \cdot n^2/2
%         \end{split}
%     \]
%
%     Thus a valid \(c\) is \(c = 2\).
%
% \end{homeworkProblem}
%
% \pagebreak
%
% \begin{homeworkProblem}
%     Let \(\Sigma = \{0, 1\}\). Construct a DFA \(A\) that recognizes the
%     language that consists of all binary numbers that can be divided by 5.
%     \\
%
%     Let the state \(q_k\) indicate the remainder of \(k\) divided by 5. For
%     example, the remainder of 2 would correlate to state \(q_2\) because \(7
%     \mod 5 = 2\).
%
%     \begin{figure}[h]
%         \centering
%         \begin{tikzpicture}[shorten >=1pt,node distance=2cm,on grid,auto]
%             \node[state, accepting, initial] (q_0)   {$q_0$};
%             \node[state] (q_1) [right=of q_0] {$q_1$};
%             \node[state] (q_2) [right=of q_1] {$q_2$};
%             \node[state] (q_3) [right=of q_2] {$q_3$};
%             \node[state] (q_4) [right=of q_3] {$q_4$};
%             \path[->]
%                 (q_0)
%                     edge [loop above] node {0} (q_0)
%                     edge node {1} (q_1)
%                 (q_1)
%                     edge node {0} (q_2)
%                     edge [bend right=-30] node {1} (q_3)
%                 (q_2)
%                     edge [bend left] node {1} (q_0)
%                     edge [bend right=-30] node {0} (q_4)
%                 (q_3)
%                     edge node {1} (q_2)
%                     edge [bend left] node {0} (q_1)
%                 (q_4)
%                     edge node {0} (q_3)
%                     edge [loop below] node {1} (q_4);
%         \end{tikzpicture}
%         \caption{DFA, \(A\), this is really beautiful, ya know?}
%         \label{fig:multiple5}
%     \end{figure}
%
%     \textbf{Justification}
%     \\
%
%     Take a given binary number, \(x\). Since there are only two inputs to our
%     state machine, \(x\) can either become \(x0\) or \(x1\). When a 0 comes
%     into the state machine, it is the same as taking the binary number and
%     multiplying it by two. When a 1 comes into the machine, it is the same as
%     multipying by two and adding one.
%     \\
%
%     Using this knowledge, we can construct a transition table that tell us
%     where to go:
%
%     \begin{table}[ht]
%         \centering
%         \begin{tabular}{c || c | c | c | c | c}
%             & \(x \mod 5 = 0\)
%             & \(x \mod 5 = 1\)
%             & \(x \mod 5 = 2\)
%             & \(x \mod 5 = 3\)
%             & \(x \mod 5 = 4\)
%             \\
%             \hline
%             \(x0\) & 0 & 2 & 4 & 1 & 3 \\
%             \(x1\) & 1 & 3 & 0 & 2 & 4 \\
%         \end{tabular}
%     \end{table}
%
%     Therefore on state \(q_0\) or (\(x \mod 5 = 0\)), a transition line should
%     go to state \(q_0\) for the input 0 and a line should go to state \(q_1\)
%     for input 1. Continuing this gives us the Figure~\ref{fig:multiple5}.
% \end{homeworkProblem}
%
% \begin{homeworkProblem}
%     Write part of \alg{Quick-Sort($list, start, end$)}
%
%     \begin{algorithm}[]
%         \begin{algorithmic}[1]
%             \Function{Quick-Sort}{$list, start, end$}
%                 \If{$start \geq end$}
%                     \State{} \Return{}
%                 \EndIf{}
%                 \State{} $mid \gets \Call{Partition}{list, start, end}$
%                 \State{} \Call{Quick-Sort}{$list, start, mid - 1$}
%                 \State{} \Call{Quick-Sort}{$list, mid + 1, end$}
%             \EndFunction{}
%         \end{algorithmic}
%         \caption{Start of QuickSort}
%     \end{algorithm}
% \end{homeworkProblem}
%
% \pagebreak
%
% \begin{homeworkProblem}
%     Suppose we would like to fit a straight line through the origin, i.e.,
%     \(Y_i = \beta_1 x_i + e_i\) with \(i = 1, \ldots, n\), \(\E [e_i] = 0\),
%     and \(\Var [e_i] = \sigma^2_e\) and \(\Cov[e_i, e_j] = 0, \forall i \neq
%     j\).
%     \\
%
%     \part
%
%     Find the least squares esimator for \(\hat{\beta_1}\) for the slope
%     \(\beta_1\).
%     \\
%
%     \solution
%
%     To find the least squares estimator, we should minimize our Residual Sum
%     of Squares, RSS:
%
%     \[
%         \begin{split}
%             RSS &= \sum_{i = 1}^{n} {(Y_i - \hat{Y_i})}^2
%             \\
%             &= \sum_{i = 1}^{n} {(Y_i - \hat{\beta_1} x_i)}^2
%         \end{split}
%     \]
%
%     By taking the partial derivative in respect to \(\hat{\beta_1}\), we get:
%
%     \[
%         \pderiv{
%             \hat{\beta_1}
%         }{RSS}
%         = -2 \sum_{i = 1}^{n} {x_i (Y_i - \hat{\beta_1} x_i)}
%         = 0
%     \]
%
%     This gives us:
%
%     \[
%         \begin{split}
%             \sum_{i = 1}^{n} {x_i (Y_i - \hat{\beta_1} x_i)}
%             &= \sum_{i = 1}^{n} {x_i Y_i} - \sum_{i = 1}^{n} \hat{\beta_1} x_i^2
%             \\
%             &= \sum_{i = 1}^{n} {x_i Y_i} - \hat{\beta_1}\sum_{i = 1}^{n} x_i^2
%         \end{split}
%     \]
%
%     Solving for \(\hat{\beta_1}\) gives the final estimator for \(\beta_1\):
%
%     \[
%         \begin{split}
%             \hat{\beta_1}
%             &= \frac{
%                 \sum {x_i Y_i}
%             }{
%                 \sum x_i^2
%             }
%         \end{split}
%     \]
%
%     \pagebreak
%
%     \part
%
%     Calculate the bias and the variance for the estimated slope
%     \(\hat{\beta_1}\).
%     \\
%
%     \solution
%
%     For the bias, we need to calculate the expected value
%     \(\E[\hat{\beta_1}]\):
%
%     \[
%         \begin{split}
%             \E[\hat{\beta_1}]
%             &= \E \left[ \frac{
%                 \sum {x_i Y_i}
%             }{
%                 \sum x_i^2
%             }\right]
%             \\
%             &= \frac{
%                 \sum {x_i \E[Y_i]}
%             }{
%                 \sum x_i^2
%             }
%             \\
%             &= \frac{
%                 \sum {x_i (\beta_1 x_i)}
%             }{
%                 \sum x_i^2
%             }
%             \\
%             &= \frac{
%                 \sum {x_i^2 \beta_1}
%             }{
%                 \sum x_i^2
%             }
%             \\
%             &= \beta_1 \frac{
%                 \sum {x_i^2 \beta_1}
%             }{
%                 \sum x_i^2
%             }
%             \\
%             &= \beta_1
%         \end{split}
%     \]
%
%     Thus since our estimator's expected value is \(\beta_1\), we can conclude
%     that the bias of our estimator is 0.
%     \\
%
%     For the variance:
%
%     \[
%         \begin{split}
%             \Var[\hat{\beta_1}]
%             &= \Var \left[ \frac{
%                 \sum {x_i Y_i}
%             }{
%                 \sum x_i^2
%             }\right]
%             \\
%             &=
%             \frac{
%                 \sum {x_i^2}
%             }{
%                 \sum x_i^2 \sum x_i^2
%             } \Var[Y_i]
%             \\
%             &=
%             \frac{
%                 \sum {x_i^2}
%             }{
%                 \sum x_i^2 \sum x_i^2
%             } \Var[Y_i]
%             \\
%             &=
%             \frac{
%                 1
%             }{
%                 \sum x_i^2
%             } \Var[Y_i]
%             \\
%             &=
%             \frac{
%                 1
%             }{
%                 \sum x_i^2
%             } \sigma^2
%             \\
%             &=
%             \frac{
%                 \sigma^2
%             }{
%                 \sum x_i^2
%             }
%         \end{split}
%     \]
%
% \end{homeworkProblem}
%
% \pagebreak
%
% \begin{homeworkProblem}
%     Prove a polynomial of degree \(k\), \(a_kn^k + a_{k - 1}n^{k - 1} + \hdots
%     + a_1n^1 + a_0n^0\) is a member of \(\Theta(n^k)\) where \(a_k \hdots a_0\)
%     are nonnegative constants.
%
%     \begin{proof}
%         To prove that \(a_kn^k + a_{k - 1}n^{k - 1} + \hdots + a_1n^1 +
%         a_0n^0\), we must show the following:
%
%         \[
%             \exists c_1 \exists c_2 \forall n \geq n_0,\ {c_1 \cdot g(n) \leq
%             f(n) \leq c_2 \cdot g(n)}
%         \]
%
%         For the first inequality, it is easy to see that it holds because no
%         matter what the constants are, \(n^k \leq a_kn^k + a_{k - 1}n^{k - 1} +
%         \hdots + a_1n^1 + a_0n^0\) even if \(c_1 = 1\) and \(n_0 = 1\).  This
%         is because \(n^k \leq c_1 \cdot a_kn^k\) for any nonnegative constant,
%         \(c_1\) and \(a_k\).
%         \\
%
%         Taking the second inequality, we prove it in the following way.
%         By summation, \(\sum\limits_{i=0}^k a_i\) will give us a new constant,
%         \(A\). By taking this value of \(A\), we can then do the following:
%
%         \[
%             \begin{split}
%                 a_kn^k + a_{k - 1}n^{k - 1} + \hdots + a_1n^1 + a_0n^0 &=
%                 \\
%                 &\leq (a_k + a_{k - 1} \hdots a_1 + a_0) \cdot n^k
%                 \\
%                 &= A \cdot n^k
%                 \\
%                 &\leq c_2 \cdot n^k
%             \end{split}
%         \]
%
%         where \(n_0 = 1\) and \(c_2 = A\). \(c_2\) is just a constant. Thus the
%         proof is complete.
%     \end{proof}
% \end{homeworkProblem}
%
% \pagebreak
%
% %
% % Non sequential homework problems
% %
%
% % Jump to problem 18
% \begin{homeworkProblem}[18]
%     Evaluate \(\sum_{k=1}^{5} k^2\) and \(\sum_{k=1}^{5} (k - 1)^2\).
% \end{homeworkProblem}
%
% % Continue counting to 19
% \begin{homeworkProblem}
%     Find the derivative of \(f(x) = x^4 + 3x^2 - 2\)
% \end{homeworkProblem}
%
% % Go back to where we left off
% \begin{homeworkProblem}[6]
%     Evaluate the integrals
%     \(\int_0^1 (1 - x^2) \dx\)
%     and
%     \(\int_1^{\infty} \frac{1}{x^2} \dx\).
% \end{homeworkProblem}

\begin{colophon}
    This document was typeset using the \LaTeXe{} document processing system
    originally developed by Leslie Lamport, based on the \TeX{} typesetting system
    created by Donald Knuth.
    The class is \texttt{latex-homework-template} by Josh Davis,
    released under the MIT license.
    All above work was done by the authors.

\end{colophon}

\end{document}
