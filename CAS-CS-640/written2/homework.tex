\documentclass[11pt,largemargins, anonymous]{homework}

\usepackage[titlesize=18pt]{colophon}

% TODO: replace these with your information
\newcommand{\hwname}{Alec Hoyland}
\newcommand{\hwemail}{U83403624}
\newcommand{\hwtype}{Homework}
\newcommand{\hwnum}{1}
\newcommand{\hwclass}{CAS CS 640}
\newcommand{\hwlecture}{0}
\newcommand{\hwsection}{A1/A5}

\begin{document}

\clearpage
\maketitle
\clearpage

\question

\begin{alphaparts}
    \questionpart
    $ \mathrm{accuracy} = (\mathrm{TP + TN}) / (\mathrm{P + N}) = (16 + 8) / (20 + 10) = 24/30 = 0.8$.

    \questionpart
    The point in ROC space is at $\left ( FP / (FP + TN) , TP / (TP + TN) \right )$.
    Therefore, $ ( 4 / (4 + 8) , 16 / (16 + 8) ) = (0.33, 0.67)$.

    \questionpart
    The F-1 score is $2 TP / (2 TP + FP + FN)$.
    Therefore, $F_1 = 2 * 16 / (2 * 16 + 4 + 2) = 0.84$
\end{alphaparts}

\question
In this neural network, the activation function is an x-shifted Heaviside function.

\begin{alphaparts}
    \questionpart
    The input vector is $[0, 0, 1, 0, 0, 1]$.
    This leads to a hidden layer state of $[2, 2]$, which is activated to
    $[1, 1]$.
    This leads to an output layer state of $[4, -4]$, which is activated to $[1, 0]$.
    Thus, the neural network concludes that Romeo and Juliet are acquaintances.
\end{alphaparts}

\begin{colophon}
    This document was typeset using the \LaTeXe{} document processing system
    originally developed by Leslie Lamport, based on the \TeX{} typesetting system
    created by Donald Knuth.
    The class is \texttt{latex-homework-class} by Jake Zimmerman,
    released under the MIT license.
    All above work was done by the authors.
\end{colophon}

\end{document}

On(x,Table, s3)∨On(x,Table,STORE(x, s3))
