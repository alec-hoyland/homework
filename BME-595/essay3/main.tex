%%%%%%%%%%%%%%%%%%%%%%%%%%%%%%%%%%%%%%%%%
% Diaz Essay
% LaTeX Template
% Version 2.0 (13/1/19)
%
% This template originates from:
% http://www.LaTeXTemplates.com
%
% Authors:
% Vel (vel@LaTeXTemplates.com)
% Nicolas Diaz (nsdiaz@uc.cl)
%
% License:
% CC BY-NC-SA 3.0 (http://creativecommons.org/licenses/by-nc-sa/3.0/)
%
%%%%%%%%%%%%%%%%%%%%%%%%%%%%%%%%%%%%%%%%%

%----------------------------------------------------------------------------------------
%	PACKAGES AND OTHER DOCUMENT CONFIGURATIONS
%----------------------------------------------------------------------------------------

\documentclass[12pt]{diazessay} % Font size (can be 10pt, 11pt or 12pt)

%----------------------------------------------------------------------------------------
%	TITLE SECTION
%----------------------------------------------------------------------------------------

\title{\textbf{A Cochlear Implant Model} \\ {\Large\itshape Essay \#3}} % Title and subtitle

\author{\textbf{Alec Hoyland} \\ \textit{BME 595}} % Author and institution

\date{\today} % Date, use \date{} for no date

%----------------------------------------------------------------------------------------

\begin{document}

\maketitle % Print the title section

%----------------------------------------------------------------------------------------
%	ABSTRACT AND KEYWORDS
%----------------------------------------------------------------------------------------

%\renewcommand{\abstractname}{Summary} % Uncomment to change the name of the abstract to something else

% \begin{abstract}
% Morbi tempor congue porta. Proin semper, leo vitae faucibus dictum, metus mauris lacinia lorem, ac congue leo felis eu turpis. Sed nec nunc pellentesque, gravida eros at, porttitor ipsum. Praesent consequat urna a lacus lobortis ultrices eget ac metus. In tempus hendrerit rhoncus. Mauris dignissim turpis id sollicitudin lacinia. Praesent libero tellus, fringilla nec ullamcorper at, ultrices id nulla. Phasellus placerat a tellus a malesuada.
% \end{abstract}

% \hspace*{3.6mm}\textit{Keywords:} lorem, ipsum, dolor, sit amet, lectus % Keywords

\vspace{30pt} % Vertical whitespace between the abstract and first section

%----------------------------------------------------------------------------------------
%	ESSAY BODY
%----------------------------------------------------------------------------------------

\subsection*{Introduction}

In this paper, we describe a model of the cochlea and auditory nerve periphery, which models the signal transduction of auditory
information from mechanical sound waves to neural signals on the auditory nerve.
Briefly, the model consists of two parts: a model of a cochlear implant, and a model of the auditory nerve.

\subsection*{The Cochlear Implant Model}

The cochlear implant model transforms a
mechanical waveform into a pulse train.

The input signal is a digitized, normalized waveform,
which passes through a filter cascade
of three gammatone filters and five low-pass Butterworth filters.
Gammatone filters have an impulse response that is the product
of a gamma distribution and a sinusoid.
The gammatone filters model the transformation of signal
due to the outer and middle ear.
The cascade of Butterworth filters captures the lower frequencies
which contain human speech information.

The real part of the signal is nonlinearly compressed by scaling
to the power of 0.9 and then is transformed into a pulse train.
The pulse train is a rectangular waveform of high and low pulses
scaled by the amplitude of the transformed input signal.
The pulse train contains information about the amplitude of the low-frequency content
of the input signal.
This models the transduction of mechanical sound waves through the basilar membrane
via activation of hair cells.
Different regions of the basilar membrane are sensitive to different frequency bands
and displacement of the membrane changes the firing rate of hair cells.
The filterbank and pulse train replaces the action of the basilar membrane and the hair cells
and provides electrical stimulation directly to the afferent fibers of the auditory nerve.

\subsection*{The Auditory Nerve Model}

The auditory nerve model consists of a population
of five leaky integrate-and-fire (LIF) neuron models.
Each LIF neuron is a ``resetting'' RC circuit
that accumulates charge and discharges in
a neural action potential (spike).
Each neuron model has different scaling and bias
parameters.
The model neurons encode physical values into spike rates.
The current injected into the neurons
increases the firing rate of the neurons,
inducing a rate code.
The encoded representation is decoded via a linear decoding matrix.

\begin{figure}[ht]
    \centering
    \includegraphics[width=\linewidth]{fig.png}
    \caption{A hastily-done diagram of the model.}
\end{figure}

\subsection*{Discussion}

This model of a cochlear implant and the auditory nerve
provides a functional and extensible model representation of signal transduction in human hearing.
The input, a sound waveform, is filtered and transformed into a pulse train,
which replaces the action of the outer, middle, and inner ear,
and provides input to the neurons of the auditory nerve.
The auditory nerve encodes the pulse train in a rate code,
which can be decoded by higher order processes using a linear map.

There are several limitations to this model.
Firstly, the cochlear implant model uses only a single electrode,
and so only captures low-frequency sound.
A more sophisticated model would use multiple channels to capture
a broader range of frequencies.
Secondly, the number of neurons in the auditory nerve model is extremely small.
A more complete model would use thousands of neurons instead of five.
The cochlear model uses a simplistic nonlinear transform to represent the stiffness
of the basilar membrane. A more accurate equation could be used.
For decoding the spiking code of the auditory nerve model,
the model uses a linear decoder.
Neural decoding is known to be nonlinear in practice,
and so a more accurate model would necessitate a nonlinear decoding algorithm,
though the linear approximation is accurate at middling ranges of activation.





% Cras gravida, est vel interdum euismod, tortor mi lobortis mi, quis adipiscing elit lacus ut orci. Phasellus nec fringilla nisi, ut vestibulum neque. Aenean non risus eu nunc accumsan condimentum at sed ipsum.
% \begin{wrapfigure}{l}{0.42\textwidth} % Inline image example, use an 'r' column type to position the figure on the right
% 	\includegraphics[width=\linewidth]{fish.png}
% 	\caption{An example fish.}
% \end{wrapfigure}
% Aliquam fringilla non diam sed varius. Suspendisse tellus felis, hendrerit non bibendum ut, adipiscing vitae diam. Lorem ipsum dolor sit amet, consectetur adipiscing elit. Nulla lobortis purus eget nisl scelerisque, commodo rhoncus lacus porta. Vestibulum vitae turpis tincidunt, varius dolor in, dictum lectus. Aenean ac ornare augue, ac facilisis purus. Sed leo lorem, molestie sit amet fermentum id, suscipit ut sem. Vestibulum orci arcu, vehicula sed tortor id, ornare dapibus lorem. Praesent aliquet iaculis lacus nec fermentum. Morbi eleifend blandit dolor, pharetra hendrerit neque ornare vel. Nulla ornare, nisl eget imperdiet ornare, libero enim interdum mi, ut lobortis quam velit bibendum nibh.

% \begin{itemize}
% 	\item First bullet point item
% 	\item Second bullet point item
% 	\item Third bullet point item
% \end{itemize}

% Morbi tempor congue porta. Proin semper, leo vitae faucibus dictum, metus mauris lacinia lorem, ac congue leo felis eu turpis. Sed nec nunc pellentesque, gravida eros at, porttitor ipsum. Praesent consequat urna a lacus lobortis ultrices eget ac metus. In tempus hendrerit rhoncus. Mauris dignissim turpis id sollicitudin lacinia. Praesent libero tellus, fringilla nec ullamcorper at, ultrices id nulla. Phasellus placerat a tellus a malesuada.

% \begin{enumerate}
% 	\item First numbered list item
% 	\item Second numbered list item
% \end{enumerate}

% %------------------------------------------------

% \section*{Conclusion}

% Fusce in nibh augue. Cum sociis natoque penatibus et magnis dis parturient montes, nascetur ridiculus mus. In dictum accumsan sapien, ut hendrerit nisi. Phasellus ut nulla mauris. Phasellus sagittis nec odio sed posuere. Vestibulum porttitor dolor quis suscipit bibendum. Mauris risus lectus, cursus vitae hendrerit posuere, congue ac est. Suspendisse commodo eu eros non cursus. Mauris ultrices venenatis dolor, sed aliquet odio tempor pellentesque. Duis ultricies, mauris id lobortis vulputate, tellus turpis eleifend elit, in gravida leo tortor ultricies est. Maecenas vitae ipsum at dui sodales condimentum a quis dui. Nam mi sapien, lobortis ac blandit eget, dignissim quis nunc.

% Donec luctus tincidunt mauris, non ultrices ligula aliquam id. Sed varius, magna a faucibus congue, arcu tellus pellentesque nisl, vel laoreet magna eros et magna. Vivamus lobortis elit eu dignissim ultrices. Fusce erat nulla, ornare at dolor quis, rhoncus venenatis velit. Donec sed elit mi. Sed semper tellus a convallis viverra. Maecenas mi lorem, placerat sit amet sem quis, adipiscing tincidunt turpis. Cras a urna et tellus dictum eleifend. Fusce dignissim lectus risus, in bibendum tortor lacinia interdum.

% \begin{table}[h] % [h] forces the table to be output where it is defined in the code (it suppresses floating)
% 	\caption{Example table.}
% 	\centering
% 	\begin{tabular}{l l r}
% 		\toprule
% 		\multicolumn{2}{c}{Name} \\
% 		\cmidrule(r){1-2}
% 		First Name & Last Name & Grade \\
% 		\midrule
% 		John & Doe & $7.5$ \\
% 		Richard & Miles & $5$ \\
% 		\bottomrule
% 	\end{tabular}
% \end{table}

% Fusce eleifend porttitor arcu, id accumsan elit pharetra eget. Mauris luctus velit sit amet est sodales rhoncus. Donec cursus suscipit justo, sed tristique ipsum fermentum nec. Ut tortor ex, ullamcorper varius congue in, efficitur a tellus. Vivamus ut rutrum nisi. Phasellus sit amet enim efficitur, aliquam nulla id, lacinia mauris. Quisque viverra libero ac magna maximus efficitur. Interdum et malesuada fames ac ante ipsum primis in faucibus. Vestibulum mollis eros in tellus fermentum, vitae tristique justo finibus. Sed quis vehicula nibh. Etiam nulla justo, pellentesque id sapien at, semper aliquam arcu. Integer at commodo arcu. Quisque dapibus ut lacus eget vulputate.

%----------------------------------------------------------------------------------------
%	BIBLIOGRAPHY
%----------------------------------------------------------------------------------------

% \bibliographystyle{IEEEtran}

% \bibliography{mybib.bib}

%----------------------------------------------------------------------------------------

\end{document}
